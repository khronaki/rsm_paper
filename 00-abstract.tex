As chip multi-processors (CMPs) are becoming more and more complex, software solutions such as parallel programming models are attracting a lot of attention.
Task-based parallel programming models offer an appealing approach to utilize complex CMPs.
However, the increasing number of cores on modern CMPs is pushing research towards the use of fine grained parallelism.
Task-based programming models need to be able to handle such workloads and offer performance and scalability.
Using specialized hardware for boosting performance of task-based programming models is a common practice in the research community.

Our paper makes the observation that task creation becomes a bottleneck when we execute fine grained parallel applications with task-based programming models.
As the number of cores increases the time spent generating the tasks of the application is becoming more critical to the entire execution.
To overcome this issue, we propose {\proposal}.
{\proposal} offers a solution for minimizing task creation overheads and relies both on the runtime system and a dedicated hardware.
%We propose \proposal, a task-based programming model that offers a minimalistic approach to runtime overheads acceleration.
On the runtime system side, {\proposal} decouples the task creation from the other runtime activities.
It then transfers this part of the runtime to a specialized hardware.
We draw the requirements for this hardware in order to boost execution of highly parallel applications.
From our evaluation using 11 parallel workloads on both symmetric and asymmetric systems, we obtain performance improvements up to 15$\times$, averaging to 3.1$\times$ over the baseline.


%In this paper we present how a specialized HW can overcome this issue by executing explicitly task creation.
%This design gives us high performance improvements for systems of up to 512 cores.
