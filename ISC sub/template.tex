%%%%%%%%%%%%%%%%%%%%%%% file template.tex %%%%%%%%%%%%%%%%%%%%%%%%%
%
% This is a general template file for the LaTeX package SVJour3
% for Springer journals.          Springer Heidelberg 2010/09/16
%
% Copy it to a new file with a new name and use it as the basis
% for your article. Delete % signs as needed.
%
% This template includes a few options for different layouts and
% content for various journals. Please consult a previous issue of
% your journal as needed.
%
%%%%%%%%%%%%%%%%%%%%%%%%%%%%%%%%%%%%%%%%%%%%%%%%%%%%%%%%%%%%%%%%%%%
%
% First comes an example EPS file -- just ignore it and
% proceed on the \documentclass line
% your LaTeX will extract the file if required

%
\RequirePackage{fix-cm}
%
%\documentclass{svjour3}                     % onecolumn (standard format)
%\documentclass[smallcondensed]{svjour3}     % onecolumn (ditto)
\documentclass[smallextended]{svjour3}       % onecolumn (second format)
%\documentclass[twocolumn]{svjour3}          % twocolumn
%
\smartqed  % flush right qed marks, e.g. at end of proof
%
\usepackage{graphicx}
%
% \usepackage{mathptmx}      % use Times fonts if available on your TeX system
%
% insert here the call for the packages your document requires
%\usepackage{latexsym}
% etc.
%
% please place your own definitions here and don't use \def but
% \newcommand{}{}
%
% Insert the name of "your journal" with
% \journalname{myjournal}
%
\begin{document}


% make the title area
\maketitle
% As a general rule, do not put math, special symbols or citations
% in the abstract
\begin{abstract}
\input{00-abstract}
\end{abstract}

% no keywords




% For peer review papers, you can put extra information on the cover
% page as needed:
% \ifCLASSOPTIONpeerreview
% \begin{center} \bfseries EDICS Category: 3-BBND \end{center}
% \fi
%
% For peerreview papers, this IEEEtran command inserts a page break and
% creates the second title. It will be ignored for other modes.

%%%%%%%%%%%%%%%%%%%
%%%%%%%%%%%%%%%%%%%
\section{Introduction}
\label{sec:intro}

%%%%%%%%%%%%%%%%%%%%%%%%%%%
%%%%%%%%%%%%%%%%%%%%%%%%%%%
Since the end of Dennard scaling~\cite{Dennard74} and the subsequent stagnation of CPU clock frequencies, computer architects and programmers rely on multi-core designs to achieve the desired performance levels.
While multi-core architectures constitute a solution to the CPU clock stagnation problem, they bring important challenges both from the hardware and software perspectives.
On the hardware side, multi-core architectures require sophisticated mechanisms in terms of coherence protocols, consistency models or deep memory hierarchies. 
Such requirements complicate the hardware design process.
On the software side, multi-core designs significantly complicate the programming burden compared to their single-core predecessors.
The different CPUs are exposed to the programmer, who has to make sure to use all of them efficiently, as well as using the memory hierarchy properly by exploiting both temporal and spatial locality.
This increasing programming complexity, also known as the Programmability Wall~\cite{Chapman2007}, has motivated the advent of sophisticated programming paradigms and runtime system software to support them.

Task-based parallelism~\cite{Blumofe:PPoPP1995, Reinders2007, Bauer2012, OmpSs} has been proposed as a solution to the Programmability Wall and, indeed, the most relevant shared memory programming standards, like OpenMP, support tasking constructs~\cite{OpenMP4.0:Manual2015}.
The task based model requires the programmer to split the code into several sequential pieces, called tasks, as well as explicitly specifying their input and output dependencies. 
%It also requires these pieces of code to be annotated in terms of input or output data dependencies.
The task-based execution model (or runtime system) consists of a master thread and several worker threads. The master thread goes over the code of the application and creates tasks once it encounters source code annotations identifying them. 
The runtime system manages the pool of all created tasks and schedules them across the threads once their input dependencies are satisfied.
To carry out the task management process, the parallel runtime system creates and maintains a Task Dependency Graph (TDG).
In this graph nodes represent tasks and edges are dependencies between them.
Once a new task is created, a new node is added to the TDG. 
The connectivity of this new node is defined by the data dependencies of the task it represents, which are explicitly specified in the application's source code.
When the execution of a task finalizes, its corresponding node is removed from the TDG, as well as its data dependencies.

This task-based runtime system constitutes of a software layer that enables parallel programmers to decouple the parallel code from the underlying parallel architecture where it is supposed to run on.
As long as the application can be decomposed into tasks, the task-based execution model is able to properly manage it across homogeneous many-core architectures or heterogeneous designs with different core types. 
A common practice in the high performance domain is to map a single thread per core, which enables the tasks running on that thread to fully use the core capacity. 
Finally, another important asset of task-based parallelism is the possibility of automatically managing executions on accelerators with different address spaces. 
Since the input and output dependencies of tasks are specified, the runtime system can automatically offload a task and its dependencies to an accelerator device (E.g. GPU) without the need for specific programmer intervention~\cite{Bueno:IPDPS2012}.
Additional optimizations in terms of software pre-fetching~\cite{Papaefstathiou2013} or more efficient coherence protocols~\cite{Manivannan2014} can also be enabled by the task-based paradigm.

Despite their advantages, task-based programming models also induce to computational costs.
For example, the process of task creation requires the traversal of several indexed tables to update the status of the parallel run by adding the new dependencies the recently created tasks bring, which produces a certain overhead.
Such overhead constitutes a significant burden, especially on architectures with several 10's or 100's of cores where tasks need to be created in a very fast rate to feed all of them.
This paper proposes the Task Generation Express ({\proposal}) approach. 
Our proposal suggests that the software and hardware are designed to eliminate the most important bottlenecks of task-based parallelism without hurting their multiple advantages. 
This paper focuses on the software part of this proposal and draws the requirements of the hardware design to achieve significant results.
%This paper proposes the Task Generation Express ({\proposal}) approach, a combined hardware-software solution to eliminate the most important bottlenecks of task-based parallelism without hurting their multiple advantages.
In particular, this paper makes the following contributions beyond the state-of-the-art:

\begin{itemize}

\item A new parallel task-based runtime system that decouples the most costly routines from the other runtime activities and thus enables them to be off-loaded to specific-purpose helper cores.

\item A detailed study of the requirements of a specific-purpose helper core able to accelerate the most time consuming runtime system activities. 
%To accelerate the time consuming runtime parts,
%\item A specific-purpose helper core able to accelerate the most time consuming runtime system activities. This hardware is also able to run user-level tasks, although it does so in a very slow way compared to a general purpose core.

\item A complete evaluation via trace-driven simulation considering 11 parallel OpenMP codes and $25$ different system configurations, including homogeneous and heterogeneous systems. %and large core counts up to $XXX$ cores. 
Our evaluation demonstrates how {\proposal} achieves average speedups of $3.1\times$ when compared against currently use state-of-the-art approaches.

\end{itemize} 

The rest of this document is organized as follows: 
Section~\ref{sec:background} describes the task-based execution model and its main bottlenecks.
Section~\ref{sec:ram} describes the new task-based runtime system this paper proposes as well as the specialized hardware that accelerates the most time-consuming runtime routines.
Section~\ref{sec:experimental} contains the experimental set-up of this paper.
Section~\ref{sec:evaluation} describes the evaluation of {\proposal} via trace-driven simulation.
Finally, Section~\ref{sec:related} discusses related work and Section~\ref{sec:conclusions} concludes this work.




%%%%%%%%%%%%%%%%%%%
%%%%%%%%%%%%%%%%%%%
\section{Background and Motivation}
\label{sec:background}
\input{02-background}

%%%%%%%%%%%%%%%%%%%
%%%%%%%%%%%%%%%%%%%
\section{Task Generation Express}
\label{sec:ram}
%We propose a hardware and software solution to overcome the performance bottleneck described in the previous section. 
%In our system we assume the existence of a specialized hardware that is responsible for the fast execution of the task creation. 
%To use such a hardware we need flexible software that manages to move the task creation to be executed there.

%The software that we employ uses a specialized queue for the task creation requests. 
%Each thread that is about to create a task instead of creating it, it inserts a task creation request in this queue. 
%The responsible hardware then reads the queue and executes the requests.

In this paper we propose a semi-centralized runtime system that dynamically separates the most computationally intensive parts of the runtime system and accelerates them on specialized hardware. 
To develop the {\proposal} we use the OpenMP programming model~\cite{OpenMP},~\cite{OpenMP4.0:Manual2013}.
The base of our implementation is the Nanos++ runtime system responsible for the parallel execution.

Nanos++ is a distributed runtime system that uses dynamic scheduling.
As most task-based programming models, Nanos++ consists of the master and the worker threads.
The master thread is launching the parallel region and creates the tasks that have been defined by the programmer{\footnote{Nanos++ also supports nested parallelism so any of the worker threads can potentially create tasks. However the majority of the existing parallel applications are not implemented using nested parallelism.}}.
The scheduler of Nanos++ consists of a~\textit{ready queue} (\textit{TaskQ}) that is shared for reading and writing among threads and is used to keep the tasks that are ready for execution.
All threads have access to the \textit{TaskQ} and once they become available they try to pop a task from the \textit{TaskQ}.
When a thread finishes a task, it performs all the essential steps described in Section~\ref{sec.background} to keep the data dependency structures consistent.
Moreover, it pushes the tasks that become ready to the \textit{TaskQ}.

%We implement the Runtime Activity Manager ({\proposal}) on top of Nanos distributed runtime system that uses dynamic scheduling~\cite.


\subsection{Implementation}

{\proposal} relieves the master and worker threads from the intensive work of task creation by offloading it on the specialized hardware.
%{\proposal} assumes the existence of a specialized hardware that accelerates this part of the runtime.
Our runtime, apart from the master and the worker threads, introduces the Special Runtime Thread (SRT). 
When the runtime system starts, it creates the SRT and binds it to the task creation accelerator, keeping its thread identifier in order to manage the usage of it.
During runtime, the master and worker threads look for ready tasks in the task ready queue and execute them along with the runtime.
Instead of querying the ready queue for tasks, the SRT looks for runtime activity requests in the Runtime Requests Queue (\textit{RRQ}) and if there are requests, it executes them.

Figure~\ref{fig:communication} shows the communication infrastructure between threads within {\proposal}.
Our system maintains two queues; the Ready Task Queue (\textit{TaskQ}) and the Runtime Requests Queue (\textit{RRQ}).
The \textit{TaskQ} is used to keep the tasks that are ready for execution. 
The \textit{RRQ} is used to keep the pending runtime activity requests. 
The master and the worker threads can push and pop tasks to and from the \textit{TaskQ} and they can also add runtime activity to the \textit{RRQ}. 
The special runtime thread (SRT) pops runtime requests from the \textit{RRQ} and executes them on the accelerator.

When the master thread encounters a task clause in the application's code, after allocating the memory needed, it calls the \texttt{createTask} as shown in Listing~\ref{taskCreation} and described in Section~\ref{sec.background}. 
{\proposal} decouples the execution of \texttt{createTask} from the master thread. 
To do so, {\proposal} implements a wrapper function that is invoked instead of \texttt{createTask}.
In this function, the runtime system checks if the SRT is enabled; if not then the default behaviour takes place, that is, to perform the creation of the task.
If the SRT is enabled, a \textit{Create} request is generated and inserted in the \textit{RRQ}.
The \textit{Create} runtime request includes the appropriate info to execute the code described in Listing~\ref{taskCreation}.
That is, the dependence analysis data, the address of the allocated task, its parent and its arguments.


\begin{lstlisting}[float, emph={void,if,return,non_critical_queue, critical_queue,not,true,and,break}, captionpos=b, caption={Pseudo-code for the SRT loop.},label=SRTloop, emph={[2]mat}, emphstyle={[2]}, aboveskip={0\baselineskip}, frame=tb, belowskip={0\baselineskip}]
1 void SRTloop() {
2  while( true ) {   
3    while(RRQ is not empty) {
4      executeRequest( RRQ.pop() );
5  }
6  if( runtime.SRTstop() ) break;
7 return; 
8}  
\end{lstlisting}

While the master and worker threads are executing tasks, the SRT is looking for \textit{Create} requests in the \textit{RRQ} to execute.
Listing~\ref{SRTloop} shows the code that the SRT is executing until the end of the parallel execution.
The special runtime thread continuously checks whether there are requests in the \textit{RRQ} (line 3). 
If there is a pending creation request, the SRT calls the \texttt{executeRequest} (line 4), which extracts the appropriate task creation data from the creation request and performs the task creation by calling the \texttt{createTask} described in Listing~\ref{taskCreation}.
%executes the task submission and inserts the task in the TDG with a call to the \texttt{executeRequest} (line 4). 
When the parallel region is over, the runtime system informs the SRT to stop execution.
This is when the SRT exits and the execution finishes (line 6).

%\kc{communication:} To relieve the master and worker threads from the intensive runtime activity we use a simple queueing communication scheme that offloads the task submission on a specialized hardware.
%Whenever a thread encounters a task submission phase, instead of executing the submit call of the scheduler, it creates a runtime request and inserts it in the RRQ. 
%Concurrently, the SRT queries the RRQ for runtime requests and as long as there are requests available it executes them.
%Moreover, the SRT executes tasks whenever there are no requests for a long time to speedup the execution.

\begin{figure}[t]%
	\centering
	\includegraphics[width=1.0\columnwidth]{figures/communication2.pdf}
	\vspace{-0.5cm}
	\caption{Communication mechanism between master/workers and SRT threads.}
	\label{fig:communication}%
	\vspace{-0.3cm}
\end{figure}






%\section{Hardware requirements}}
%\label{sec:hw}
\input{hw}
%%%%%%%%%%%%%%%%%%%
%%%%%%%%%%%%%%%%%%%
\section{Experimental Methodology}
\label{sec:experimental}
\input{04-experimental}

%%%%%%%%%%%%%%%%%%%
%%%%%%%%%%%%%%%%%%%
\section{Evaluation}
\label{sec:evaluation}
\input{05-evaluation}

%%%%%%%%%%%%%%%%%%%%
%%%%%%%%%%%%%%%%%%%%
\section{Related Work}
\label{sec:related}
\input{06-related}

%%%%%%%%%%%%%%%%%%%
%%%%%%%%%%%%%%%%%%%
\section{Conclusions}
\label{sec:conclusions}
\input{07-conclusions}

%%%%%%%%% -- BIB STYLE AND FILE -- %%%%%%%%
\bibliographystyle{abbrv}
\bibliography{references}

%\end{thebibliography}

%\bibitem{IEEEhowto:kopka}
%H.~Kopka and P.~W. Daly, \emph{A Guide to \LaTeX}, 3rd~ed.\hskip 1em plus
%  0.5em minus 0.4em\relax Harlow, England: Addison-Wesley, 1999.

\end{document}
% end of file template.tex

